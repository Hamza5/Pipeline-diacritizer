\documentclass[a4paper,12pt]{report}
\usepackage[utf8]{inputenc} % Encodage du fichier
\usepackage[T1]{fontenc} % Encodage des fonts nécessaire pour le Latin
\usepackage[english]{babel} % Pour changer la langue des mots générés et choisir la bonne mise en page
\usepackage{lmodern} % Le latin modèrne
\usepackage[top=2cm, bottom=2cm, left=3cm, right=2cm]{geometry} % Définir les marges de la page
\usepackage[hidelinks,
            unicode=true,
            pdftitle={Automatic Arabic diacritization using deep learning},
            pdfauthor={Hamza ABBAD}
            pdfdisplaydoctitle=true]{hyperref} % Pour les liens
\usepackage{fancyhdr} % Pour le style de la page
\usepackage[font=it]{caption} % Rendre les titres des tableaux italiques
\usepackage{microtype} % Pour éviter les underfuls et les overfuls
\usepackage{graphicx} % Pour les images
\usepackage{subcaption} % Pour mettre plusieurs images sur la même ligne
\usepackage{float} % Pour empêcher le déplacement des tableaux et des figures.
\usepackage{babelbib} % Pour changer la langue dans la bibliographie
\usepackage{amsmath} % Pour des fonctions mathématiques
%\usepackage[onelanguage,longend,boxruled,algoruled,linesnumbered,algochapter]{algorithm2e} % Pour les algorithmes
%\usepackage{listings} % Pour le code source et le contenu d'un fichier
\usepackage{array}

\graphicspath{ {pictures/} } % Spécifier le répertoire contenant les images
\DisableLigatures[f]{encoding=*}
\renewcommand \thechapter{\Roman{chapter}} % Utiliser les numéros romans pour les chapitres
\captionsetup{labelfont=it,textfont=it,labelsep=period} % Changer le style des légendes
%\SetAlgoCaptionSeparator{\unskip.} % pour les légendes d'algorithmes
%\AtBeginDocument{ % Changer les légendes
%  \renewcommand\tablename{\itshape Tableau}
%  \renewcommand{\figurename}{\itshape Figure}
%	% Renommer la table des matières
%	\renewcommand{\contentsname}{Sommaire}
%}


\date{}
% Style de l'entête et le pied de la page
\setlength{\headheight}{16pt}
\pagestyle{fancy}
\fancyhead[L]{} % Enlever la section
\fancyhead[R]{\footnotesize\slshape{\nouppercase{\leftmark}}} % Titre du chapitre en minuscule avec taille 10
\fancyfoot[C]{}
\fancyfoot[R]{\thepage} % Déplacer le numéro de la page vers la droite de la page

\fancypagestyle{plain}{
\renewcommand{\headrulewidth}{0pt}
\fancyhf{}
\fancyfoot[R]{\thepage}
}

% Espace entre les lignes
\linespread{1.3}

% Paramètres du code
%\lstset{numbers=left,captionpos=b,frame=single,breaklines,basicstyle=\ttfamily\footnotesize}

% Code pris de https://tex.stackexchange.com/a/95616/109916 et corrigé
% Début
\makeatletter
\newcommand{\emptypage}[1]{
  \cleardoublepage
  \begingroup
  \let\ps@plain\ps@empty
  \pagestyle{empty}
  #1
  \cleardoublepage
  \endgroup}
\makeatother
% Fin

\begin{document}

% \renewcommand{\sfdefault}{phv}
%\include{Page_de_garde}
%\emptypage{
%{
%\setlength{\parskip}{0.7em plus 0.2em minus 0.2em}
%\include{Remerciements}
%}
%\tableofcontents
%\listoffigures
%\listoftables
%}

\setlength{\parskip}{0.6em plus 0.1em minus 0.1em}

% Changer le style des listes de descriptions
\let\olddescription\description
\let\endolddescription\enddescription
\renewenvironment{description}{
\let\olditem\item
\renewcommand\item[2][]{\olditem[\textit{##1}] ##2}
\begin{olddescription}}
{\end{olddescription}\ignorespacesafterend}

% Redéfinition des chapitres et sections pour les inclure dans le sommaire
\makeatletter
\let\oldchapter\chapter
\newcommand{\@chapterstar}[1]{\cleardoublepage\phantomsection\addcontentsline{toc}{chapter}{#1}{\oldchapter*{#1}}\markboth{#1}{}}
\newcommand{\@chapternostar}[1]{{\oldchapter{#1}}}
\renewcommand{\chapter}{\@ifstar{\@chapterstar}{\@chapternostar}}
\let\oldsection\section
\newcommand{\@sectionstar}[1]{\phantomsection\addcontentsline{toc}{section}{#1}{\oldsection*{#1}}}
\newcommand{\@sectionnostar}[1]{{\oldsection{#1}}}
\renewcommand\section{\@ifstar{\@sectionstar}{\@sectionnostar}}
\makeatother

% Mes commandes
\newcommand{\keyword}[1]{\emph{#1}}

\setcounter{page}{1}
%\include{Introduction}
% Changer les captions pour cacher les références dans la table des figures
%\let\oldcaption\caption
%\renewcommand{\caption}[2]{\oldcaption[#1]{#1~#2}}
\chapter{Natural Language Processing for Arabic}
%\include{Chapitre2}
%\let\caption\oldcaption
%\include{Chapitre3}
%\include{Chapitre4}
%\include{Chapitre5}
%\include{Conclusion}

\bibliographystyle{babplain}
\renewcommand\btxauthorcolon{.}
\emptypage{
\bibliography{references}
}

\let\section\oldsection % pour éviter que le résumé soit visible dans le sommaire comme une section
%\include{Resume}
\end{document}
